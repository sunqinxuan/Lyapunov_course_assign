\documentclass[12pt,a4paper]{article}

\usepackage[pdftex]{graphicx}
%\usepackage{cite}
\usepackage{indentfirst}
\setlength{\parindent}{1em}
\usepackage{enumerate}
\usepackage{geometry}
\geometry{left=1in,right=1in,top=1in,bottom=1in}
%\usepackage{times}
%\usepackage{mathptmx}
%\usepackage{listings}
%\usepackage[framed,numbered,autolinebreaks,useliterate]{mcode}
\usepackage{amsmath}
\usepackage{amssymb}
\usepackage{tcolorbox}

\title{Lyapunov based Nonlinear Control - Assignment3}
\author{Sun Qinxuan}

\begin{document}
\maketitle

\section*{Problem 1}

\indent Find the equilibrium point(s) of the following friction system
$$\dot{v}+2av|v|+bv=c,a>0,b>0,c>0$$
and analyze the corresponding stability.

\subsection*{Solution 1}

\indent Rewrite the system equation as follows
\begin{equation}
\dot{v}=-2av|v|-bv+c.
\end{equation}
Solve (\ref{eq1}) to get the equilibrium points.
\begin{equation}
-2av|v|-bv+c=0 .
\label{eq1}
\end{equation}
Considering the absolute value of $v$, (\ref{eq1}) turns out to be in the form
\begin{equation}
\left\{
    \begin{aligned}
    &-2av^2-bv+c=0 &v\ge 0\\
    &2av^2-bv+c=0 &v<0
    \end{aligned}
\right. .
\label{eq2}
\end{equation}
Solve (\ref{eq2}) and we know that the system has only one equilibrium point
\begin{equation}
v_s=\frac{b-\sqrt{b^2+8ac}}{-4a}.
\label{vs}
\end{equation}

\indent To analyze the stability on this equilibrium point, we first define the translation
\begin{equation}
x\triangleq v-v_s,
\end{equation}
then for the system $\dot{x}=-a(x+v_s)|x+v_s|-b(x+v_s)+c$, it has one equilibrium point $x_s=0$. Likewise, the system equation can be rewritten as
\begin{equation}
\dot{x}=
\left\{
    \begin{aligned}
    &-2a(x+v_s)^2-b(x+v_s)+c &x\ge -v_s\\
    &2a(x+v_s)^2-b(x+v_s)+c &x<-v_s
    \end{aligned}
\right. .
\label{eq3}
\end{equation}

\indent Choose the Lyapunov function
\begin{equation}
V(x)=\frac{1}{2}x^2.
\end{equation}
Obviously, $V(x)$ is positive definite and goes to infinity when $x\rightarrow\infty$. Then
\begin{equation}
\dot{V}(x)=x\dot{x}=
\left\{
    \begin{aligned}
    &-2ax(x+v_s)^2-bx(x+v_s)+cx &x\ge -v_s\\
    &2ax(x+v_s)^2-bx(x+v_s)+cx &x<-v_s
    \end{aligned}
\right. .
\label{vdot}
\end{equation}

\indent We can tell from (\ref{vdot}) that when $x<-v_s$, each item of $\dot{V}(x)$ is non-positive. And they don't take the value of zero at the same $x$. So $\dot{V}(x)$ is negative definite when $x<-v_s$. As for the case of $x\ge -v_s$, known that $-2av_s^2-bv_s+c=0$, $\dot{V}(x)$ can be written as follows
\begin{equation}
\dot{V}(x)=-x^2[2a(x+2v_s)+b], x\ge -v_s.
\label{vdot2}
\end{equation}
It can be seen from (\ref{vdot2}) that when $x\ge -v_s$, $\dot{V}(x)\le 0$ holds true. And $\dot{V}(x)=0$ if and only if $x=0$. So $\dot{V}(x)$ is also negative definite. Therefore, the equilibrium point $v=v_s$ is globally asymptotically stable.

\section*{Problem 2}

\indent For the following nonlinear system
$$\dot{x}=-kx+\sin^3(x)+x\cos^2(x)$$
where $k>2$ denotes a positive constant. Find its equilibrium point and use Lyapunov method to analyze its stability (get a conclusion as strong as possible).

\begin{itemize}
    \item Show that there is {\bf only} one equilibrium point at
        $$x=0.$$
    \item Utilize the Lyapunov method to analyze its stability around origin.
\end{itemize}

\subsection*{Solution 2}

\subsubsection*{(1) Show that there is {\bf only} one equilibrium point at $x=0$.}

\indent It's obvious that when $x=0$, $\dot{x}=0$. So $x=0$ is one of the system's equilibrium points.

\indent In what follows, we'll prove that $x=0$ is the only equilibrium point.

\indent When $x>0$, the system equation can be rewritten as
\begin{equation}
\dot{x}=(1-k)x+\sin^2(x)(\sin(x)-x)
\end{equation}
%where
%\begin{equation}
%\begin{aligned}
%&f_1(x)=(1-k)x\\
%&f_2(x)=\sin^2(x)(\sin(x)-x).
%\end{aligned}
%\end{equation}
Let
\begin{equation}
f(x)=\sin(x)-x.
\end{equation}
Then take the derivative of $f(x)$ w.r.t. $x$.
\begin{equation}
\frac{{\rm d}f(x)}{{\rm d}x}=\cos(x)-1<0.
\label{dfx}
\end{equation}
From (\ref{dfx}) we can see that $f(x)$ is decreasing for $x\in {\mathbb R}$. Since $f(0)=0$, $f(x)$ satisfied
\begin{equation}
f(x)
\left\{
    \begin{aligned}
    &>0 &x<0\\
    &=0 &x=0\\
    &<0 &x>0
    \end{aligned}
\right. .
\label{fx}
\end{equation}
And also $(1-k)x<0, \sin^2(x)>0$. As a result, $\dot{x}<0$ holds true when $x>0$. Likewise, when $x<0$ we can deduce that $\dot{x}>0$.

\indent From all above, we can conclude that
\begin{equation}
\dot{x}
\left\{
    \begin{aligned}
    &>0 &x<0\\
    &=0 &x=0\\
    &<0 &x>0
    \end{aligned}
\right. .
\label{xdot}
\end{equation}
So the system has only one equilibrium point $x=0$.

\subsubsection*{(2) Utilize the Lyapunov method to analyze its stability around origin.}

\indent Choose the Lyapunov function
\begin{equation}
V(x)=\frac{1}{2}x^2.
\end{equation}
Obviously, $V(x)$ is positive definite and goes to infinity when $x\rightarrow\infty$. Then
\begin{equation}
\dot{V}(x)=x\dot{x}.
\end{equation}
As shown from (\ref{xdot}), $\dot{V}(x)$ satisfied $\dot{V}(x)\le 0$. And $\dot{V}(x)=0$ holds true if and only if $x=0$. So $\dot{V}(x)$ is negative definite. Therefore, the equilibrium point $x=0$ is globally asymptotically stable.

\indent Furthermore,
\begin{equation}
\dot{V}(x)=x\dot{x}=(-k+1)x^2+x\sin^2(x)(sin(x)-x).
\end{equation}
From (\ref{fx}), we know that for $x\in{\mathbb R}$, $x\sin^2(x)(sin(x)-x)\le 0$. Therefore, $\dot{V}(x)$ satisfies
\begin{equation}
\dot{V}(x)\le -(k-1)x^2=-2(k-1)V(x).
\end{equation}

\begin{tcolorbox}[width=\textwidth]
\begin{quote}
{\bf Lemma 8}

if $V(t)>0$ and $\dot{V}(t)\le -\gamma V(t)$, where $\gamma$ is a positive constant, then
$$ V(t)\le V(0)d^{-\gamma t}. $$
\end{quote}
\end{tcolorbox}
As known from Lemma 8,
\begin{equation}
V(t)\le V(t_0)e^{-2(k-1)(t-t_0)}.
\end{equation}
As a result,
\begin{equation}
x(t)\le x(t_0)e^{-(k-1)(t-t_0)}.
\end{equation}
According to the definition of {\bf Exponentially Stable}, the equilibrium point $x=0$ is also globally exponentially stable.

\end{document} 